
\pagenumbering{arabic}
\section{Introduction}

\subsection{Background}

The emergence of deep learning, a subset of artificial intelligence, has revolutionized image analysis and pattern recognition tasks. Deep learning models, particularly convolutional neural networks (CNNs), have demonstrated remarkable success in various domains, including medical image analysis, object detection, and plant disease identification. CNNs can automatically learn intricate features from images, enabling them to distinguish between healthy and diseased plants with high accuracy.

In the context of sunflower disease identification, several deep learning models have been explored. Inception-V2, a well-established CNN architecture, has shown promise in image classification tasks due to its efficient use of computational resources and ability to capture complex features. U-Net, another CNN architecture designed for image segmentation, excels at delineating objects or regions within images, making it suitable for localizing disease lesions within plant leaves.

However, individual models may have limitations. Inception-V2, while effective in classification, may not provide precise localization of disease symptoms. U-Net, although capable of segmentation, may not fully leverage the global context of the image for accurate classification. Therefore, a hybrid approach that combines the strengths of both models, namely Inception-V2 for classification and U-Net for segmentation, holds the potential to enhance the overall accuracy and robustness of sunflower disease identification.
\clearpage
\subsection{Problem Statement}

\justifying{
The accurate and timely identification of sunflower diseases remains a significant challenge in modern agriculture, hindering efficient disease management and yield optimization.\par
Sunflower diseases, caused by various pathogens like fungi, bacteria, and viruses, manifest through diverse visual symptoms on leaves, stems, and other plant parts. Early detection and identification of these diseases are crucial for implementing effective control measures and mitigating yield losses. However, traditional manual inspection methods are time-consuming, labor-intensive, and prone to human error. Moreover, the subtle and often overlapping nature of disease symptoms makes accurate diagnosis difficult, even for experienced experts. The lack of reliable, automated, and accessible disease identification tools further exacerbates the problem, particularly in resource-limited agricultural settings.
}

\subsection{Objective}
To develop and evaluate a hybrid deep learning architecture that combines the classification capabilities of Inception-V2 with the segmentation prowess of U-Net to achieve superior accuracy and precision in identifying and localizing sunflower diseases from leaf images

\clearpage

\subsection{Project Features}
\subsubsection{abc}
\blindtext
\subsubsection{bcd}
\blindtext
\subsubsection{cde}
\blindtext
\subsubsection{efg}
\blindtext

\subsection{Feasibility analysis}
\blindtext
\subsubsection{Economic Feasibility}
\blindtext

\subsubsection{Technical Feasibility}
\blindtext

\subsubsection{Legal Feasibility}
\blindtext

\subsubsection{Operational Feasibility}
\blindtext


\newpage




